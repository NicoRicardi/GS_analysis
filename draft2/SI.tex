\documentclass[amsmath,amssymb,preprint,aip,jcp]{revtex4-1}
\usepackage[table]{xcolor}
\usepackage{amsmath}
\usepackage{amsfonts}
\usepackage{graphicx}
\usepackage{float}
\usepackage{silence}
\WarningFilter{revtex4-1}{Repair the float}

\begin{document}
\author{Niccol\`{o} Ricardi}
\email{Niccolo.Ricardi@unige.ch}
\author{Cristina E. Gonz\'{a}lez-Espinoza}
\email{Cristina.GonzalezEspinoza@unige.ch}
\author{Tomasz Adam Weso\l{}owski}
\email{Tomasz.Wesolowski@unige.ch}
\affiliation{Department of Physical Chemistry, University of Geneva, Geneva (Switzerland)}
% 
\date{\today}
\title{Supplementary Material for: Negativity of the target density in practical Frozen-Density Embedding Theory based calculations}

\maketitle
\section{Appendix: boundaries of $P[\rho^{o}_A + \rho_B - \rho_{AB}^o]$}
The parameter $P$ is bound by:
\begin{equation} \label{eq:P_bound}
 M[\rho_{B} - \rho^{o}_{AB}] \leq P[\rho_A^o+\rho_B - \rho_{AB}^{o}] \leq % \int \rho_{AB}^{o} 
 N_{AB}.
\end{equation}
Due to the fact that:
\begin{equation}
 \int \rho^{o}_A + \rho_B = \int \rho_{AB}^o = N_{AB},
\end{equation}
where $N_{AB}$ is the number of electrons of the supersystem, the integrated difference of these two densities is zero:
\begin{align}
 & \int \rho^{o}_A + \rho_B - \rho_{AB}^o = 0 \\ \nonumber
 & \int\limits_{\rho_{AB}^o <\rho^{o}_A + \rho_B}\rho^{o}_A + \rho_B - \rho_{AB}^o + \int\limits_{\rho_{AB}^o > \rho^{o}_A + \rho_B}\rho^{o}_A + \rho_B - \rho_{AB}^o = 0 \\ \nonumber
 & \int\limits_{\rho_{AB}^o < \rho^{o}_A + \rho_B}\rho^{o}_A + \rho_B - \rho_{AB}^o = \int\limits_{\rho_{AB}^o > \rho^{o}_A + \rho_B}\rho_{AB}^o - \rho^{o}_A - \rho_B 
\end{align}
As a consequence, we can reformulate $P[\rho^{o}_A + \rho_B - \rho_{AB}^o]$:
\begin{align}\label{eq:P_alternatives}
P[\rho^{o}_A + \rho_B - \rho_{AB}^o] = & \frac{1}{2} \cdot \int \vert \rho^{o}_A + \rho_B - \rho_{AB}^o \vert \\ \nonumber
P[\rho^{o}_A + \rho_B - \rho_{AB}^o] = & \frac{1}{2} \cdot \int\limits_{\rho_{AB}^o <\rho^{o}_A + \rho_B}\rho^{o}_A + \rho_B - \rho_{AB}^o + \\ \nonumber
 & \frac{1}{2} \cdot \int\limits_{\rho_{AB}^o > \rho^{o}_A + \rho_B}\rho^{o}_A + \rho_B - \rho_{AB}^o \\ \nonumber
P[\rho^{o}_A + \rho_B - \rho_{AB}^o] = & \int\limits_{\rho_{AB}^o <\rho^{o}_A + \rho_B}\rho^{o}_A + \rho_B - \rho_{AB}^o.
\end{align}
We can then split the integration space and obtain:
\begin{align}
P[\rho^{o}_A + \rho_B - \rho_{AB}^o] = & \int\limits_{\rho_{AB}^o < \rho_B}\rho^{o}_A + \rho_B - \rho_{AB}^o + \\ \nonumber
 & \int\limits_{\rho_B \leq \rho_{AB}^o < \rho_B + \rho^{o}_A}\rho^{o}_A + \rho_B - \rho_{AB}^o.
\end{align}
Which in turn leads to:
\begin{align}\label{eq:P_M+terms}
 P[\rho^{o}_A + \rho_B - \rho_{AB}^o] =  &M[\rho_{AB}^o - \rho_B] + \int\limits_{\rho_{AB}^o < \rho_B}\rho^{o}_A + \\ \nonumber
 & \int\limits_{\rho_B \leq \rho_{AB}^o < \rho_B + \rho^{o}_A}\rho^{o}_A + \rho_B - \rho_{AB}^o.
\end{align}
The fact that both integrals on the right hand side of Eq.~\ref{eq:P_M+terms} are non-negative guarantees that:
\begin{equation}
 P[\rho^{o}_A + \rho_B - \rho_{AB}^o] \geq M[\rho_{AB}^o - \rho_B],
\end{equation}
while the upper bound in Eq.~\ref{eq:P_bound} is apparent from Eq.~\ref{eq:P_alternatives}.

\end{document}
